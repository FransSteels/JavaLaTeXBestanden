\documentclass{beamer}

\usepackage{verbatim}

\usetheme{CambridgeUS}

\usecolortheme{dolphin}

%\setbeamertemplate{footline}[page number]

\setbeamertemplate{navigation symbols}{}

\title{JUnit}
\author{Java Cursisten}
\institute{INTEC Brussel}
\date{\today}


\begin{document}


\begin{frame}

\titlepage

\end{frame}


\begin{frame}

\frametitle{Overzicht}
\tableofcontents

\end{frame}


\section{Inleiding}


\begin{frame}

\frametitle{We hebben alles al getest...}

{\Large Alle code wordt getest\\~\\
Nadat we een stukje code geschreven hebben runnen we het programma en doen onze \textit{'acceptance test'}.\\~\\
\textit{'We code, we compile, we run and we test... over and over again.'}}

\end{frame}


\begin{frame}

\frametitle{Wat is JUnit}

\begin{itemize}

\item {\LARGE JUnit is een framework}
\item {\LARGE de facto standaard voor het testen van Java applicaties}
\item {\LARGE Wij gebruiken versie 4.x}

\end{itemize}

\end{frame}

\begin{frame}

\frametitle{Framework?}

{\Large Een framework is een semi-complete applicatie.\\~\\
Het biedt een herbruikbare, algemene structuur die gedeeld wordt tussen applicaties.\\~\\
Ontwikkelaars incorporeren zo een framework in hun applicatie en passen het aan aan de
specifieke noden van hun applicatie.}

\end{frame}


\begin{frame}

\frametitle{Testen en testen zijn drie...}

\begin{enumerate}

\item {\Large Unit test: een afgescheiden eenheid werk (veelal \'e\'en methode) }
\item {\Large Functional test: een bepaalde functionaliteit (samenwerking tussen units)}
\item {\Large Integration test: een heel systeem (samenwerking tussen \textbf{meer} units)}
  
\end{enumerate}

\end{frame}


\begin{frame}

\frametitle{Waarom JUnit gebruiken?}

\begin{itemize}

\item {\LARGE Geautomatiseerde testen voor een ge\"isoleerd stukje code}
\item {\LARGE Controleren of de code de verwachte output geeft bij een bepaalde input}
\item {\LARGE Ze 'verzekeren' dat de nieuwe code werkt}
\item {\LARGE Ze 'verzekeren' dat de oude code werkt}
\item {\LARGE Debuggen is makkelijker}
\item {\LARGE Ze bieden een soort documentatie aan}

\end{itemize}

\end{frame}


\begin{frame}

\frametitle{Robert C Martin}

\begin{center}
{\huge \textit{'Any program feature without an automated test doesn't exist.'}}
\end{center}

% \begin{figure}
% 
% \includegraphics[width=100pt]{RobertMartin.jpg}
% 
% \label{Robert C Martin}
% 
% \end{figure}

\end{frame}


\section{JUnit in Eclipse}

\begin{frame}

\frametitle{JUnit in Eclipse}

\begin{center}
{\huge DEMO}
\end{center}

\end{frame}


\begin{frame}

\frametitle{opdracht}

\begin{center}
{\Large Maak een calculator applicatie met methodes voor}
\end{center}

\begin{itemize}
\item {\Large add}
\item {\Large substract}
\item {\Large multiply}
\item {\Large divide}
\item {\Large root}
\end{itemize}

\begin{center}
{\Large Maak ook een test voor elke methode want...}
\end{center}

% \begin{figure}
% 
% \includegraphics[width=60pt]{RobertMartin.png}
% 
% \end{figure}

\begin{center}
\textit{{\LARGE 'We will not ship shit!'}}
\end{center}

\end{frame}


\section{The life and death of the Tests}


\begin{frame}

\frametitle{The life and death of the Tests}

{\Large In JUnit zit een \textit{testrunner} die al de testen zal uitvoeren.\\~\\
De testrunner gaat op zoek naar al de methodes met de annotatie \textbf{'@Test'}.\\~\\
Voor iedere testmethode in een klasse wordt er een nieuw object van die klasse 
ge\"instantieerd zodat de testmethodes geen invloed op elkaar kunnen uitoefenen.}

\end{frame}

\begin{frame}

\frametitle{Reducing boilerplate code}

{\Large Met \textbf{'@Before'} en \textbf{'@After'} kan je methodes annoteren die dan voor of na de elke test
in de klasse worden uitgevoerd.\\~\\

Met \textbf{'@BeforeClass'} en \textbf{'@AfterClass'} kan je statische methoden annoteren die voor en
na de reeks testen in de klasse worden uitgevoerd.\\~\\

Met \textbf{'@Ignore'} kan je tijdelijk een test uitschakelen.}

\end{frame}

\begin{frame}

\frametitle{\textbf{@Before @After @BeforeClass @AfterClass @Ignore}}

\begin{center}
{\huge DEMO}
\end{center}

\end{frame}


\section{Assert ! Assert ! Assert ?}


\begin{frame}

\frametitle{Assert... \href{http://junit.sourceforge.net/javadoc/}{\textbf{{\small Klik hier voor de JUnit API}}}}

\begin{itemize}

  \item {\Large assertEquals ( type expected, type actual )}
  \item {\Large assertTrue ( boolean condition )}
  \item {\Large assertNull ( Object obj )}
  \item {\Large assertNotNull ( Object obj )}
  \item {\Large assertSame ( Object expected, Object actual )}
  \item {\Large assertNotSame ( Object expected, Object actual )}
   
\end{itemize}

\end{frame}


\end{document}